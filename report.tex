\documentclass[a4paper,12pt,twoside,english]{article}

\usepackage{times}  
\usepackage{amsmath}
\usepackage{amssymb}
\usepackage{graphicx}
\usepackage{wrapfig}
\usepackage{theorem}
\usepackage[latin1]{inputenc}
\usepackage{latexsym}
\usepackage{url}
\usepackage[english]{babel}
\usepackage{subcaption}
\usepackage{caption}

\headsep          0.25in
\headheight       12pt
\setlength{\topmargin}{-\headsep}
\addtolength{\topmargin}{-\headheight}
\evensidemargin 0in
\setlength{\textwidth}{\paperwidth}
\addtolength{\textwidth}{-2.0in}
\setlength{\textheight}{\paperheight}
\addtolength{\textheight}{-2.0in}
\addtolength{\textheight}{-1\topskip}
\divide\textheight  by \baselineskip
\multiply\textheight  by \baselineskip
\addtolength{\textheight}{\topskip}



\title{Angry Cuts}
\author{Anton Alb\`{e}rt Karlstr\"{o}m, Lovisa Dahl, \\Emma Edvardsson
 and Hannes Ingelhag \\Norrk\"{o}ping}

%Gör så att alla formler blir lite större
\everymath{\displaystyle}

\begin{document}

%\pagestyle{empty}

%----------------------------- Front matter of the document

\begin{figure}
\begin{center}
\includegraphics[width=6cm]{bilder/LiTH_sigill_col.png} 
\end{center}
\end{figure}

\maketitle
\pagenumbering{gobble}

\newpage
\begin{abstract}
This report covers the development of defining a model for a physical system and to simulate the dynamics of it. In this project the physical system consists of several weights, each attached to a pendulum, which will be released from their rope at the same time and converted into projectiles. The weights will interact with both static objects, the walls and floor of the room, and dynamic objects, other weights inside the room until they run out of velocity. The development starts with a prestudy in MATLAB, to refine the formulas and implementation before it is transfered into C++ and OpenGL. The simulation is performed by using numerical integration approximations, both Euler's explicit method and Runge-Kutta 4 is used and evaluated. 
\vfill
\end{abstract}
\newpage
\tableofcontents  % chapter with the table of contents
\addtocontents{toc}{\protect\thispagestyle{empty}}

%----------------------------- Body of the document
\newpage

\pagenumbering{arabic}
\pagestyle{plain}


\setcounter{page}{1}
\section{Introduction}
\subsection{Background}
Angry Cuts is created as a project to simulate a two dimensional scene of a weight attached to a pendulum which will be converted into a projectile. 
The final concept is a simulation of several weights. Initially is each attached to a pendulum, and later all ropes are being cut at the same time which results in a projectile movement for all the weights. In this state the weights are interacting with both the walls and each other.

Figure 1 shows the first idea, where a pendulum is supposed to hit a tower of wooden bars. The projectile movement was inspired by the slingshot aiming towards a given target in the game {\itshape Angry birds}. The idea of releasing a weight from a pendulum was claimed from {\itshape Cut the rope}. In order to create a scene with real physics, the collisions in the scene require a careful calculation of the collisions, both with static and dynamic objects. Since objects in equilibrium are difficult to handle in a simulation, the idea was modified and extended with several balls instead of the wooden bars. The definitive idea is illustrated in Fig. 2, the scene consists of a pendulum in a room, which later will be released as a projectile.


\begin{figure}[h]
\includegraphics[height=3.5cm]{bilder/ideasketch.png}
\centering
\caption{Original idea}
\end{figure}
\begin{figure}[h]
\includegraphics[height=4cm]{bilder/Matlab_Pendulum1.png}
\centering
\caption{Modified idea}
\end{figure} 



\subsection{Purpose}
The purpose of this report is to describe and follow the development of the scene with pendulums, projectiles and collisions in two dimensions. The scene uses physically correct models and formulas for collisions with static and dynamic objects. Air resistance and energy loss are taken into consideration in the pendulum, the projectile and the collision.

Another purpose with this project is to use Matlab to test and develop the model before transferring the model into C++ and OpenGL. 

\section{Physical model}
The first part of this project was to find the physical description of the first two parts of the model: the pendulum and the projectile movement. The second part was to implement collision detector and air resistance. 
\subsection{Pendulum}
A physical model of a nonlinear pendulum (1)\footnote{See \cite[p.~64]{Hal:10}} was required since the system operates with large angles. \cite{Nor:06} \\ \\
$\frac{d^2\theta}{dt^2} + \frac{g}{L}*sin(\theta(t)) = 0$ \hfill (1) \\  

Figure 3 illustrates the model of the pendulum, $g$ represents the gravity constant, $L$ the length of the rope of the pendulum and $\theta$ is the angle of the pendulum perpendicular to the gravity.  $\frac{d^2\theta}{dt^2}$ is the angular acceleration.

 \begin{table}[h!]
  \centering
  \begin{tabular}{c  c}
        \begin{minipage}{0.5\textwidth}
      \includegraphics[height=60mm]{bilder/pendulum_2.png}
      \centering
%      \captionsetup{justification=raggedright, singlelinecheck=false}
      \captionof{figure}{Physical model of a pendulum}
    \end{minipage}
    & 
  \begin{minipage}{0.5\textwidth}
      \includegraphics[height=60mm]{bilder/projectile_dynamics_new.png}
     \centering
      \captionsetup{justification=raggedright, singlelinecheck=false}
      \captionof{figure}{Initial dynamics of the projectile}
    \end{minipage} \\
  \end{tabular}
\end{table}

\subsection{Projectile}
The outcoming velocity from the pendulum was captured as the initial velocity for the projectile movement. In order to transfer the movement into projectile, illustrated in Fig. 4, the velocity was split into horizontal (2) and vertical velocity (3). The angle ${\theta_i}$ between the velocity vector and the horizontal axis determined the magnitude of $v_x(t)$ and $v_y(t)$. The acceleration (4) was also derived from this method and the vertical acceleration was affected by gravity. 
 \\ \\
$v_x(t) = v_{out}(t) * cos(\theta_i(t))$ \hfill (2) \\
$v_y(t) = v_{out}(t) * sin(\theta_i(t))$ \hfill (3) \\
$a_y(t) = a_{out}(t) * sin(\theta_i(t)) - g$ \hfill (4) \\


 
\subsection{Air resistance}
To add air resistance to the model, the velocity should be reduced according to the angle perpendicular to the gravity. The air resistance (5) was calculated using the density of air ${\rho}$, the drag coefficient of a sphere $c_d$, the cross sectional area {\itshape A}  of the object and the velocity $v$ of the weight.
The air resistance was added to affect the velocity decreasingly which was performed by appending the $F_d$ to $v_y$ in (3). The values used  could be found in table 1, Appendix 1. \\ \\
$F_d = \frac{1}{2}( \rho * v^2(t) * c_{d} * A)$ \hfill (5) 

\subsection{Collision detection}
\subsubsection{Collision with walls and floor}
The second part of the model was to consider the collisions with the static objects in the room. This was performed by reflecting the direction after the collision according to the normal of the surface hit by the weight. The velocity was reduced with 50\%  to simulate an energy loss from the collision.  Equation 6 and 7 were used to simulate a collision between the left wall and our weight. In equation 7 the $x_{weight}$ represent the horizontal position of the weight, $x_{wall}$ the horizontal position of the left wall and $r$ the weights radius.  \\ \\
$v_x= -0.5*v_x$ \hfill (6) \\
$x_{weight} = x_{wall}+ r$ \hfill (7)

\subsubsection{Collision between objects}
The third part of the physical system was to handle collision between two moving objects. In this project all collisions are considered elastic, which means that the objects cannot stick together in the collision.
A collision is detected when the distance between two weights is smaller than the sum of the radii. The distances can be calculated using Pythagorean theorem, illustrated in Fig. 5. 
Since a numerical method is used to calculate the next step of the scene, the weights might overlap, step 1 Fig. 6, and due to that get stuck to each other, an unwanted phenomena. To get a more realistic simulation of the collision, the boundaries of the weights must intersect instead of overlap. To handle this scenario, the  distance between the two colliding objects was calculated to find out if they overlap. If the weights overlap, following steps were performed according to Fig. 6. In the first step the normal plane and the collision plane was calculated, the second step was to move the objects to their previous positions. Step 3 was to move the objects towards each other in the direction of the normal plane, until they intersect. The normal plane can be determined as the distance between center points of the weights. The collision plane was found as the orthogonal to the normal plane, as illustrated in Fig. 7. 

\begin{table}[h!]
  \centering
  \begin{tabular}{c c}
        \begin{minipage}{0.5\textwidth}
      \includegraphics[height=3.1cm]{bilder/pythagora.png} 
      \centering
   \captionof{figure}{Distance between two weights\\ using the pythagorean theorem \\} 
    \end{minipage} 
 &
  \begin{minipage}{0.5\textwidth}
      \includegraphics[height=3.2cm]{bilder/Collision_management.png}
      \centering
     \captionof{figure}{\\ \textit{Step 1:} Overlapping weights \\ \textit{Step 2}: Collision and normal planes are found \\ \textit{Step 3:} Collision can now be calculated}
    \end{minipage} 
    \end{tabular}
\end{table}

The initial velocity vectors could now be described in the coordinate plane consisting of the normal- and the collision plane, according to Fig 8. These vectors can be obtained by the operation vector dot product. When these vectors are found the new velocity vectors after the collision can be determined according to (8) and (9).  \\ \\
$ v_{1a}(t) = \frac{1}{m_1+m_2}*(v_1(t)(m_1-m_2) + 2m_2*v_2(t)) $ \hfill (8) \\ \\
$ v_{2a}(t) = \frac{1}{m_1+m_2}*(v_2(t)(m_2-m_1) + 2m_1*v_1(t))$ \hfill (9) \\ \\

$v_{1a}(t)$ and $v_{2a}(t)$ denotes the velocity of the weights, after the collision, $m_1$ and $m_2$ describes the masses of the two colliding weights. $v_1$ and $v_2$ are the velocities of the weights before the collision. 

\begin{table}[h!]
  \begin{tabular}{c c}
    \begin{minipage}{0.5\textwidth}
      \includegraphics[height=3cm]{bilder/collisionplane.png}
      \centering
      \captionof{figure}{Collision plane}
    \end{minipage}
    &
  \begin{minipage}{0.5\textwidth}
      \includegraphics[height=3cm]{bilder/newvelocitiesvectors_new.png}
      \centering
      \captionof{figure}{Modified idea}
    \end{minipage} \\
  \end{tabular}
\end{table}

\subsection{Numerical implementation}
In many cases it could be hard to calculate the equations analytically, a well known method to solve this is to use numerical approximations. There are many different numerical methods, some are very simple and others complex and more accurate. In this project two different methods was implemented, Euler's explicit method and the Runge-Kutta 4.
\subsubsection{Euler's explicit method}
Euler$'$s explicit method (10) is used to calculate the vertical and horizontal positions of the weight and the angular velocity continuously, over time of the nonlinear pendulum. \\ \\
$y(t + h) = y(t) + f(t,y(t))*h$  \hfill (10) \\

In (10), $h$ denotes the step length, $y(t)$ is the function and $f$ describes the derivative of $y$.
\subsubsection{Runge-Kutta method}
Runge-Kutta\footnote{See \cite[p.~460, 491]{Mat:04}} is another numerical method. Compared to Euler it gives a more stable system and can be used with a larger step size. The most common form of the Runge-Kutta is RK4, a fourth order method, which is described in (11).\\ \\
$k_1 = h*f(x_i, y_i) $ \\ \
$k_2 = h*f(x_i + 0.5*h, y_i + 0.5*k_1) $ \\
$k_3 = h*f(x_i + 0.5*h, y_i + 0.5*k_2)$ \hfill (11) \\
$k_4 = h*f(x_i + h, y_i + k_3)$ \\

By using these four equations the next step can be calculated according to (12). \\ \\
$ y(i+1) = y(i) + \frac{1}{6}( k_1 + 2*k_2 + 2*k_3 + k_4) $ \hfill (12)

\subsubsection{Implementation of numerical methods}

Figure 9 and 10 illustrates the implementation of the two numerical integration methods used in this project.

\begin{table}[h!]
  \centering
  \begin{tabular}{c c}
      \begin{minipage}{0.5\textwidth}
      \vspace{1.2cm}
      \includegraphics[width=\linewidth]{bilder/eulerCode.png}
      \centering
      \captionsetup{justification=centering, singlelinecheck=false}
      \captionof{figure}{Implementation of Euler$'$s \\explicit method}
    \end{minipage}
    &
  \begin{minipage}{0.5\textwidth}
      \includegraphics[width=\linewidth]{bilder/rungeKuttaCode.png}
    %  \centering
      \captionsetup{justification=centering, singlelinecheck=false}
      \captionof{figure}{Implementation of \\Runge-Kutta}
    \end{minipage} \\
  \end{tabular}
\end{table}

\section{Simulation}
\subsection{Graphical implementation}
\subsubsection{MATLAB}
A physical model of one weight was defined and implemented in MATLAB for a simulation test. Constants for the calculations can be found in \textit{Appendix 1}. This was made iteratively, starting with the pendulum alone followed by adding the projectile movement and the last step with collision detection with static objects. The final simulation consisted of a weight attached to a pendulum, see Fig. 11, which was released after a given time and converted into a projectile. The weight would bounce on the walls and the ground until the added energy loss caused it to stop moving. 

The second part was to add interaction between two weights in the same room. This step required a fully working model of one weight and an additional collision management to handle collision between two dynamic objects. Fig. 12 shows two weights of different radii.

\begin{table}[h!]
  \centering
  \begin{tabular}{c  c}
        \begin{minipage}{0.5\textwidth}
      \includegraphics[width=\linewidth, width=60mm]{bilder/Matlab_Pendulum1.png}
      \centering
     % \captionsetup{justification=raggedright, singlelinecheck=false}
      \captionof{figure}{Plot of one pendulum}
    \end{minipage}
    & 
  \begin{minipage}{0.5\textwidth}
      \includegraphics[width=\linewidth, width=60mm]{bilder/Matlab_Pendulum2.png}
%      \centering
      \captionsetup{justification=raggedright, singlelinecheck=false}
      \captionof{figure}{Plot of two pendulums}
    \end{minipage} \\
  \end{tabular}
\end{table}



\subsubsection{OpenGL 3.3}
To improve the graphical representation of the model the code was converted to C++ and OpenGL. The software used in the project were Sublime Text 2, Visual Studio and CMake. 

A possibility to change the number of weights in the scene was implemented, the pendulums were drawn with even intervals according to the number of weights. Fig. 13 and Fig. 14 presents the scene with three weights in OpenGL. The constants used for these figures are the same as previous.

\begin{table}[h!]
  \centering
   \begin{tabular}{c c}
      \begin{minipage}{0.5\textwidth}
      \includegraphics[width=\linewidth, width=60mm]{bilder/OpenGL_pendulum0.png}
      \centering
      \captionof{figure}{Simulation of three pendulums, first version}
   \end{minipage}
    & 
   \begin{minipage}{0.5\textwidth}
   \vspace{1.3cm}
     \includegraphics[width=\linewidth, width=60mm]{bilder/OpenGL_bounce0.png}
     \centering
     \captionof{figure}{One projectile, first version}
    \end{minipage} \\
  \end{tabular}
\end{table}

\section{Result}
The result retrieved from this project was a program where the user could enter the desired number of pendulums in a GUI, see Fig.15. The trigger of movement for the pendulums were their initial angle given by the user or in the program code. During the execution of the program the user could follow the flow of energy in the scene. The weights were positioned depending on the given number and were set to a random color. In Fig. 16 and Fig. 17 seven weights were implemented.

\begin{table}[h!]
  \centering
   \begin{tabular}{c c}
    \multicolumn{2}{c}
      {\begin{minipage}{0.5\textwidth}
      \includegraphics[width=\linewidth, width=60mm]{bilder/ConsoleAndProgram.png}
      \centering
      \captionof{figure}{Seven weights simulated in OpenGL}
    \end{minipage} } 
      \\
     \begin{minipage}{0.5\textwidth}
      \includegraphics[width=\linewidth, width=60mm]{bilder/OpenGL_pendulum1.png}
      \centering
      \captionof{figure}{Seven weights simulated in OpenGL}
    \end{minipage} 
    & 
    \begin{minipage}{0.5\textwidth}
      \includegraphics[width=\linewidth, width=60mm]{bilder/OpenGL_bounce1.png}
      \centering
      \captionof{figure}{Seven bouncing weights in a room}
    \end{minipage} 
  \end{tabular}
\end{table}
In the first part, the pendulums were given an initial angle and were affected by air resistance. In this scene the balls have the properties of lead,, and the air resistance coefficient was calculated from the circular shape.  Values of the constants could be found in table 1, Appendix 1.


\section{Discussion}
The implementation of the physical equations into MATLAB started as soon as the idea was found.
The work was most of the time divided into groups of two and these pairs implemented and discussed code together which made the working hours efficient.

During the conversion to C++, a couple of obstacles in terms of different operative systems and the different versions of OpenGL caused problems. These problems took much time and focus from the project itself. OpenGL 3.3 has some new implementations for drawing objects, different from what we have used before, which was the biggest problem.

The Euler's explicit method was used as the first numerical integration method. However, Euler gave a larger numerical error which grows for each iteration. This was noticeable since the simulation got unstable when using a large step size. To minimize this problem the Runge-Kutta 4 method was implemented, which represents a more accurate model. This method considers four coherent steps, each dependent on the previous one, from which a weighted average representing the next step of the slope could be calculated. 

Another difficulty with the simulation was to handle time as a parameter. In this project the simulation was performed as fast as the current computer could compute the code, which brought troubles with different computers. Some computers calculated the code too fast and in that case the simulation could look unrealistic.

One idea was to implement a simple model so that the balls could roll off each other to avoid the scenario that weights gets stuck on top of each other at low velocities. However, this was not implemented due to lack of time. 

\newpage
\bibliography{angrybib}
\bibliographystyle{plain}

\newpage
\pagenumbering{gobble}
\section*{Appendix 1}


\begin{center}
\begin{table}[!ht]
\captionsetup{justification=raggedright, singlelinecheck=false}
\caption{Constants for the weights}
	\begin{tabular}{| l r | l  c |}
		\hline
		\multicolumn{4}{| c |}{CONSTANTS} \\ \hline
		Radius & r & $0.02$ &$(m)$ \\ \hline
		Volume & V & $1.68*10^{-5}$ & $(m^{3})$ \\ \hline
		Density & $\rho$  & $11340$ & $(kg/m^{3})$ \\ \hline
		Mass & m & $0.19$ & $(kg)$ \\ \hline
		Air Coefficient & $c_d$  & $0.47$ & \\ \hline
		Density of air & $\rho$  & $1.225$ & $(kg/m^{3})$ \\ \hline
	\end{tabular}
\end{table}
\end{center}

\end{document}

